% Background section
\section{Background and Related Work}

\subsection{Prompt Injection and Adversarial Prompts}

The security of Large Language Models (LLMs) has become a significant focus due to vulnerabilities arising from prompt-based attacks. These attacks are analogous to classic software injection exploits, occurring within natural language input rather than code input~\cite{arthur2024promptinjection, lakera2024promptguide}. Initially, attacks were straightforward jailbreak prompts, instructing the model explicitly to ignore safety constraints~\cite{arthur2024promptinjection}. However, by mid-2023, researchers identified several sophisticated, automated strategies, including adversarial suffixes~\cite{zou2023universal} and multi-turn dialogue manipulations to incrementally override the model's defenses~\cite{zou2024oneshot}. Prompt attack methods subsequently evolved into indirect methods, such as covert injections hidden in content read by models from external sources, demonstrating that any input data could be weaponized~\cite{lakera2024promptguide, venturebeat2024prompt}. Techniques like obfuscation, token smuggling, payload splitting, and encoding malicious instructions became prevalent, bypassing simplistic input filters~\cite{learnprompting2024obfuscation}. Anthropic's discovery of many-shot jailbreaking (conditioning models by long-context demonstrations of malicious behaviors) further exploited weaknesses introduced by extended context capabilities in advanced LLMs~\cite{anthropic2024manyshot}.

\subsection{Existing Defense Mechanisms}
Current defenses against prompt injection can be categorized into four primary approaches. Static Analysis approaches employ rule-based filters and regex patterns that inspect input prompts for known attack patterns~\cite{nvidia2023nemo}. Dynamic Analysis involves runtime monitoring systems that analyze model behavior during execution~\cite{inan2023llm}. Model-based Defenses include techniques like RLHF and constitutional AI that align models to reject harmful instructions~\cite{bai2022constitutional}. System-level Protection encompasses input sanitization, API rate limiting, and access control mechanisms~\cite{owasp2024top10llm}.

Despite these defenses, recent research has demonstrated their limitations. Most static defenses fail against obfuscation techniques, while model-based defenses can be circumvented through careful prompt engineering~\cite{zou2023universal}.

\subsection{Self-Adaptive Attack Generation Techniques}

Red-teaming methodologies have evolved rapidly, transitioning from manual techniques to automated, AI-assisted tools. State-of-the-art approaches span several methodologies: Gradient-based Optimization approaches like GCG~\cite{zou2023universal} leverage gradient information to optimize adversarial suffixes; Evolutionary methods such as GBDA~\cite{guo2021gradient} generate and evolve adversarial prompts through mutation and selection; Reinforcement Learning techniques like AutoDAN~\cite{liu2023autodan} employ RL to optimize attack strategies based on model feedback; and Multi-agent Systems such as RedAgent~\cite{xu2024redagent} use collaborative frameworks to discover novel attack vectors.

These approaches have demonstrated success in discovering vulnerabilities but fail to adapt to evolving model defenses. Additionally, they lack the ability to leverage semantic relationships between attack strategies.

\subsection{Retrieval-Augmented Generation and Graph-RAG}

Retrieval-Augmented Generation (RAG) enhances language model performance by incorporating external information via retrieval mechanisms. While early RAG approaches relied on vector-based similarity search~\cite{lewis2020retrieval}, recent advances have explored graph structures to better capture entity relationships.

GraphRAG extends traditional RAG by organizing knowledge in graph form, enabling complex reasoning and relationship traversal. Key works in this area include Edge et al.~\cite{edge2024local}, who proposed a hierarchical graph approach for query-focused summarization; Peng et al.~\cite{peng2024graph}, who formalized GraphRAG components including graph-guided retrieval and graph-enhanced generation; and Han et al.~\cite{han2024retrieval}, who integrated heterogeneous data to improve retrieval precision.

Our work differs from these approaches by applying GraphRAG specifically to adversarial prompt generation, leveraging graph structures to represent relationships between attack techniques.

\subsection{Theoretical Framework}
We formalize the adaptive attack generation problem as follows:

Let \(G = (V, E)\) be a knowledge graph where \(V\) represents the set of nodes for prompts, techniques, and vulnerabilities; \(E\) represents the set of edges for relationships and transitions; \(f: V \times V \rightarrow \mathbb{R}\) is the edge weight function based on attack success; and \(\phi: V \rightarrow \mathbb{R}^d\) is the node embedding function.

The attack generation problem can be formulated as finding a path \(P = (v_1, ..., v_k)\) that maximizes:
\[
\text{score}(P) = \sum_{i=1}^{k-1} f(v_i, v_{i+1}) + \lambda \cdot \text{diversity}(P)
\]
where \(\text{diversity}(P)\) measures technique variation and \(\lambda\) balances exploitation vs. exploration.

This formulation allows us to apply graph traversal algorithms to systematically explore the space of potential attacks while adapting to model responses. 